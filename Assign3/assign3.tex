\documentclass[a4paper, 14pt]{article}
\usepackage{graphicx}
\usepackage{fancyhdr}
\newcommand{\HRule}{\rule{\linewidth}{0.5mm}}
\title{\bf Assignment 03}
\author{Xugang Zhou \\ Fangzhou Yang \\ Alexander Graf}
\pagestyle{fancy}
\lhead{{\bf Distributed Algorithm}\\WS 2013/14, Prof. Dr.-Ing. Jan Richling}
\rhead{Assignmen 02}
\renewcommand{\headrulewidth}{0.4pt}

\renewcommand{\theenumi}{\alph{enumi}}

\begin{document}
\begin{titlepage}
\begin{center}
\vfill
\textsc{\LARGE Distributed Algorithm}\\[1.5cm]
\textsc{\Large }\\[0.5cm]

\HRule \\[0.4cm]
{\huge \bfseries Assignment 03}\\[0.4cm]
\HRule \\[1.5cm]
\begin{minipage}{0.4\textwidth}
\begin{flushleft} \large
\large{\textbf{Group 11}}
\end{flushleft}
\end{minipage}
\begin{minipage}{0.4\textwidth}
\begin{flushright} \large
\begin{tabular}{ll}
Xugang \textsc{Zhou} & 352032\\
Fangzhou \textsc{Yang} & 352040\\
Yuwen \textsc{Chen} & 352038
\end{tabular}
\end{flushright}
\end{minipage}
\vfill
{\large \today}\\
\end{center}
\end{titlepage}
\thispagestyle{fancy}

\section{Election}
\begin{enumerate}
\item To guarantee the correctness of the election when a leaf node is an initiator, an extension should be add to the algorithm:\\
 \emph{Leafs that are initiators send immediately after the explosion message also their contraction message with their ID}. \\
Consider a special scenario where one leaf initiator $a$ owns the maximum id and it's connected to node $b$. During initialization $a$ sends an explorer to $b$, if it is the first explorer arriving at $b$, $b$ would not forward any further explorer from other nodes to $a$, since it's already explored. In that case $a$ would never receive an explorer, and thus would never announce its own id to the network. This leads to an incorrect election result. After adding the extension, $a$ get a chance to forward its own id and the algorithm can still work properly.

\item The algorithm won't work on a non-tree topology, since it strongly depends on the contraction messages sent by leaves. As an example, in a ring topology there wouldn't be any leaf, which means no contraction message can be generated. In such a case the algorithm terminates after exploring phase and never enters contraction phase.

\end{enumerate}

\section{Termination}
This algorithm can work properly.\\
Proof: \\
\begin{itemize}
	\item $Terminated \Rightarrow S = R$:\\
if a termination is detected, which means all processes are passive and no messages are on the way at a this time point. Thus all sent messages are received. Therefore, $S = R$.\\
	\item $S = R \Rightarrow Terminated$:\\
Assume the system is not terminated, then there is at least one message still one the way, or there is at least one active process. In Atom Model, actions are atomic and need no time, once a process receives a message, it can send out messages instantly. Therefore there is at least one message, which go through the boundary between the past and the future. Because each message has its own message id, assume the message's is x, then $x \in S $ and $x \notin R$, which is contradict to $S = R$. Therefore $$S = R \Rightarrow Terminated$$\\
\end{itemize}

\end{document}

