\documentclass[a4paper, 14pt]{article}
\usepackage{graphicx}
\usepackage{fancyhdr}
\usepackage{algorithm}
\usepackage{algorithmicx}
\usepackage{algpseudocode}
\usepackage{amsmath}
\newcommand{\HRule}{\rule{\linewidth}{0.5mm}}
\title{\bf Assignment 09}
\author{Xugang Zhou \\ Fangzhou Yang \\ Yuwen Chen}
\pagestyle{fancy}
\lhead{{\bf Distributed Algorithms}\\WS 2013/14, Prof. Dr.-Ing. Jan Richling}
\rhead{Assignmen 09}
\renewcommand{\headrulewidth}{0.4pt}
\begin{document}
\begin{titlepage}
\begin{center}
\vfill
\textsc{\LARGE Distributed Algorithms}\\[1.5cm]
\textsc{\Large }\\[0.5cm]

\HRule \\[0.4cm]
{\huge \bfseries Assignment 09}\\[0.4cm]
\HRule \\[1.5cm]
\begin{minipage}{0.4\textwidth}
\begin{flushleft} \large
\large{\textbf{Group 11}}
\end{flushleft}
\end{minipage}
\begin{minipage}{0.4\textwidth}
\begin{flushright} \large
\begin{tabular}{ll}
Xugang \textsc{Zhou} & 352032\\
Fangzhou \textsc{Yang} & 352040\\
Yuwen \textsc{Chen} & 352038
\end{tabular}
\end{flushright}
\end{minipage}
\vfill
{\large \today}\\
\end{center}
\end{titlepage}
\thispagestyle{fancy}

\section{Self-Stabilizing Spanning Tree Theory}
\begin{itemize}
\item a.) if the root fails or the root are not reachable, another spanning tree with a different root node forms, else If other links or nodes fails, another spanning tree with the same root node forms; if local data structures corrupts, the node may declare itself as a root, but it will be suppressed after it receives the correct heart beat message from the original root, thus the spanning tree will stay the same; if message are lost, a timeout will occur, and the value of $P_f$ will be decreased. If $P_f \le 0$, this node will declare itself as a root, but it will also be suppressed after it receives the correct heart beat message from the original root, thus the spanning tree will stay the same, too.

\item b.) If the topologies are changed in the self-stabilizing spanning tree, a new tree will be formed with a different root, because the root must always be the node with smallest ID.

\item c.) We can use a shortest Path Algorithm to rank the topologies, so that the constructed self-stabilizing spanning tree can always have a good performance.

\end{itemize}


%\newpage
%\bibliographystyle{plain}
%\bibliography{gc}{}
\end{document} 